\documentclass{article}%
\usepackage{amsmath}
\usepackage{graphicx}
\usepackage{amsfonts}%
\usepackage{amssymb}
\usepackage{geometry}
\geometry{a4paper, scale = 0.8}

\begin{document}
\begin{center}
\textbf{Homework 1}\bigskip
\end{center}
\begin{enumerate}
\item Exercise 2.1 

\textbf{Proof.} Use induction on $k$. When $ k = 2$, by definition of convex, it is trivial. Suppose for some $k \le n$ the proposition holds, i.e., 
$\sum_{i=1}^{n} \theta_{i}x_i\in C $, then 
$$
\sum_{i=1}^{n+1}\theta_{i}x_i = \sum_{i=1}^{n-1}\theta_ix_i + (1-\sum_{i=1}^{n-1}\theta_i)\left(\frac{\theta_n}{\theta_{n}+\theta_{n+1}}x_n + \frac{\theta_{n+1}}{\theta_{n}+\theta_{n+1}}x_{n+1}\right),
$$
and by definition $\hat{x_n} = \frac{\theta_n}{\theta_{n}+\theta_{n+1}}x_n + \frac{\theta_{n+1}}{\theta_{n}+\theta_{n+1}}x_{n+1} \in C$, hence by induction the sum is also in $C$.

\item Exercise 2.2

\textbf{Proof.} First, if $C$ is convex, since each line $l$ is convex, $C\cap l$ is convex. Conversely, for any two points $x, y\in C$, let $l$ be the line crossing both $x$ and $y$, then for all $\theta_1, \theta_2 $ satisfying the conditions, $\theta_1x+\theta_2y\in l\cap C\subset C $. Hence $C$ is convex.

If $C$ is affine, then for each line $l$, $x_1, x_2\in C\cap l $, and $\theta\in \mathbb{R}$, $x = \theta x_1 + (1-\theta)x_2 = x_2 + \theta(x_1 - x_2) \in l$, and by definition $x\in C$. Hence $x \in C\cap l$. The converse case is just the same with convex.

\item Exercise 2.5

\textbf{Sol. } $d = \frac{|b_1-b_2|}{\lVert a\rVert_2}$.

\item Exercise 2.7

\newcommand{\lv}{\lVert}
\newcommand{\rv}{\rVert}
\textbf{Sol.} Notice 
$$
\begin{aligned}
\lv x-a \rv_2 - &\lv x-b \rv_2 = ((x-b) + (b-a))^\top((x-b) + (b-a)) - (x-b)^\top (x-b)\\
&= (x-b)^\top(b-a)+(b-a)^\top(x-b)+(b-a)^\top(b-a) \\
&= 2(b-a)^\top(x-b)+(b-a)^\top(b-a)\\
&= 2(b-a)^\top (x-\frac{a+b}{2}),
\end{aligned}
$$
Hence the set can be written as $\{x\mid(b-a)^\top (x-\frac{a+b}{2}) \le 0\}$.

\item Exercise 2.8

\textbf{(a)} $S$ is a polyhedra. It can be regarded as a parallelogram spanned by $a_1 $ and $a_2 $. $S$ can be regraded as a intersect of a hyperplane and four halfspaces:

1. The plane $S_0 $ spanned by $a_1 $, $a_2 $, which can be written as $A^\top x = 0 $, with $r(A) = n-2$.

2. The twin halfspaces parallel with $a_1 $ and perpendicular to $S_0 $, which an be represented as $\{x+y_1a_1+y_2a_2\mid a_1^\top x = a_2^\top x = 0, -1\le y_2 \le 1\}$.

3. The twin halfspaces parallel with $a_2 $ and perpendicular to $S_0 $, which can be represented as $\{x+y_1a_1+y_2a_2\mid a_1^\top x = a_2^\top x = 0, -1\le y_1 \le 1\}$.

\textbf{(b)} $S$ is a polyhedra (trivial by definition).

\textbf{(c)} $S$ is the intersection of $R_{+}^{n} $ and closed unit ball, hence it cannot be described by finite number of linear inequalities.

\textbf{(d)} $S$ is just $\{x\mid \lv x \rv_{\infty} \le 1 \}$, which can be described by the intersection of $n$ hyperplanes.

\item Exercise 2.11

\textbf{Proof.} Pick $x = (x_1, x_2)^\top $ and $y = (y_1, y_2)^\top $ from $S$, then by Jensen's inequality,
$$
(\theta x_1+(1-\theta)y_1)(\theta x_2+(1-\theta)y_2) \ge (x_1x_2)^\theta(y_1y_2)^{1-\theta} \ge 1.
$$
Hence $S$ is convex. The generalization can be shown by induction, and the details is just the same as problem 1.

\item Exercise 2.12

\textbf{(a)} Convex, polyhedron.

\textbf{(b)} Convex, polyhedron.

\textbf{(c)} Convex, polyhedron.

\textbf{(d)} Convex, halfspace.

\textbf{(e)} uncertain. 

\textbf{(f)} Convex.

\textbf{(g)} not convex, it is a ball.

\item Exercise 2.15

\textbf{(a)} Convex

\textbf{(b)} Convex

\textbf{(c)} Convex

\textbf{(d)} Convex

\textbf{(e)} Convex

\textbf{(f)} It is equivalent to
$$
\sum_{i=1}^{n}a_i^2p_i - \left(\sum_{i=1}^{n}a_ip_i\right)^2 \le \alpha, 
$$
which is not convex. 

\textbf{(g)} same as (f).

\end{enumerate}
\end{document}